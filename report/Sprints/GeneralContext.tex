\chapter{General Context }

\newpage

\lhead{\leftmark}

\cfoot{\thepage}

\parindent=0.5in


\section{Introduction}
We will commence our report with the first chapter, dedicated to introducing the project context. Firstly, we will present the host environment, followed by a general Presentation of the project. We will then propose a solution based on a comprehensive analysis of the existing situation. Finally, we will outline our project management methodology and its fundamental principles. This chapter will provide an overview of the report and establish a framework for the subsequent sections.

\section{Host environment Presentation}
In progress...
\section{Presentation of the project}
Agriculture plays a crucial role in the economy of many countries, and leveraging technology can substantially enhance its efficiency, productivity, and profitability. However, the agriculture sector faces significant challenges, including:
\begin{itemize}
    \item Inefficient Processes: Manual and time-consuming agricultural processes can lead to inefficiencies, increased costs, and reduced productivity.

    \item Supply Chain Complexity: Agricultural supply chains can be complicated, with various factors such as weather, transportation, and storage conditions impacting product quality and delivery timelines.
\end{itemize}

To address these challenges, an Agriculture ERP solution can automate and streamline manual processes, improve supply chain visibility, and enable data-driven decision-making to enhance productivity, reduce costs, and increase profitability. 
\section{Study of the existing situation}
\subsection{Presentation of the study}
In today's world, many farmers continue to manage their operations using manual methods such as spreadsheets, paper records, or basic accounting software, as they may not have access to specialized agriculture ERP systems. However, some farmers may opt to use generic ERP systems that are not designed to cater to the unique needs of the agriculture industry. These systems may not provide the necessary features and functionalities required to manage agricultural processes effectively, such as tracking crop growth, managing soil quality, or monitoring weather conditions. As a result, these limitations could lead to inefficiencies, increased costs, and reduced productivity for farmers who rely on these systems.
\subsection{Criticism of the existing situation }
Many farmers still rely on standalone ERP systems that are not built to cater to their specific needs like JD EDWARDS, SAP, SAGE, … etc.
 These ERP systems can be prohibitively expensive and lack the necessary features for effective farm management. Additionally, the integration of sensors into these systems is often limited, and most sensors have their own applications, making it difficult to integrate them with an ERP system. This limits the ability of farmers to collect and analyse data in real-time, hindering their decision-making and potentially impacting their profitability.
\section{Proposed solution}
The solution to improve agriculture productivity is to build a specific ERP system that caters to the needs of farmers. AgroSense is an ERP system that has been designed to handle all agriculture-related modules. It is a flexible system that can be used on any device such as smartphones, tablets, and laptops. The system is fast and provides quick replies to farmers' needs. One of the key features of AgroSense is the integration of real-time sensors data in the production line. This allows for the monitoring and collection of real-time data about crop growth, weather conditions, soil quality, water levels, and other factors. This data can be used to make more informed decisions about planting, harvesting, irrigation, and fertilization. Furthermore, AgroSense integrates AI technologies such as machine learning algorithms and predictive analytics. Machine learning algorithms can be used to analyse weather data to predict optimal planting times or to identify pest infestations early. Predictive analytics can be used to forecast crop yields and market prices, enabling farmers to make informed decisions about planting and selling their products. By adopting AgroSense, farmers can benefit from a more efficient and data-driven approach to agriculture, which can significantly improve productivity, reduce costs, and increase profitability.
\section{Methodology Used}

\subsection{Selection of the method}

SCRUM is a popular framework for managing complex software development projects. It is based on the principles of Agile methodology and places a strong emphasis on collaboration, flexibility, and continuous improvement. One of the primary benefits of using SCRUM is its ability to manage changing requirements and priorities throughout the development process. This is achieved through the use of iterative, time-boxed sprints, where the team works together to deliver a potentially shippable product increment at the end of each sprint.

SCRUM also promotes close collaboration between team members, including the Product Owner, SCRUM Master, and Development Team. This ensures that everyone is aligned with the project's objectives and working towards a common goal. Furthermore, the SCRUM framework facilitates communication and transparency throughout the development process, enabling the team to identify and address any issues or roadblocks in a timely manner.

Finally, SCRUM emphasizes continuous improvement through regular retrospectives and reviews, where the team reflects on their progress and identifies opportunities for further improvement. This enables the team to continuously optimize their processes and practices, leading to greater productivity and efficiency.

\subsection{SCRUM roles}
\vspace{0.2cm}

\begin{itemize}

    \item\textbf{Product Owner} : Me
    
    \vspace{0.2cm}
    
    \item\textbf{Scrum Master} : ?
    
    \vspace{0.2cm}
    
    \item\textbf{Development team} : Me
    
\end{itemize}
\section{Conclusion}
